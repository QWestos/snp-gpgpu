\documentclass[runningheads]{llncs}

\setcounter{page}{1}
\authorrunning{X. Zeng et al.}
\titlerunning{Matrix Representation of Spiking Neural P Systems}

\usepackage{amsmath,amssymb,latexsym,graphicx,url}
\usepackage {graphicx}
%\usepackage {todonotes,ulem}

%\def\l{[\ }
%\def\r#1{{{\ ]_{}}_{}}_{#1}}
%\newcommand{\memb}[2]{[\, {#1} \,]_{{#2}}}
%\newcommand{\emptymemb}[1]{[\;\;]_{{#1}}}
%\newcommand{\ang}[1]{\langle {#1} \rangle}
%\newcommand{\Lra}{\Longrightarrow}
%\newcommand{\lra}{\longrightarrow}
%\newcommand{\ra}{\rightarrow}
%\newcommand{\qed}{\hspace*{1mm}\hfill$\Box$

\begin{document}

\title{Matrix Representation of Spiking Neural P Systems}

\author{Xiangxiang Zeng\inst{1}, Henry Adorna\inst{2}, Miguel \'{A}ngel
Mart\'{i}nez-del-Amor\inst{3},\\
Linqiang Pan\inst{1,}\thanks {Corresponding author. Tel.:
+86-27-87556070; Fax: +86-27-87543130.}, Mario J.
P\'{e}rez-Jim\'{e}nez\inst{3}}
\institute{Image Processing and Intelligent Control Key Laboratory of Education Ministry \\
Department of Control Science and Engineering\\
Huazhong University of Science and Technology\\
Wuhan 430074, Hubei, China \\
{\tt xzeng@foxmail.com, lqpan@mail.hust.edu.cn} \and
Department of Computer Science\\
(Algorithms and Complexity)\\
University of the Philippines\\
Diliman 1101 Quezon City, Philippines\\
{\tt hnadorna@dcs.upd.edu.ph}\and
Research Group on Natural Computing \\
Department of Computer Science and Artificial Intelligence\\
University of Sevilla\\
Avda. Reina Mercedes s/n, 41012 Sevilla, Spain\\
{\tt \{perezh,marper\}@us.es }}

\maketitle

\begin{abstract}

Spiking neural P systems (SN P systems, for short) are a class of
distributed parallel computing devices inspired from the way neurons
communicate by means of spikes. In this work, a discrete structure
representation of SN P systems with extended rules and without delay is proposed.
Specifically, matrices
are used to represent SN P systems. In order to represent the
computations of SN P systems by matrices, configuration vectors are
defined to monitor the number of spikes in each neuron at any given
configuration; transition net gain vectors are also introduced to
quantify the total amount of spikes consumed and produced after the
chosen rules are applied. Nondeterminism of the systems is assured
by a set of spiking transition vectors that could be used at any
given time during the computation. With such matrix representation,
it is quite convenient to determine the next configuration from a
given configuration, since it involves only multiplication and
addition of matrices after deciding the spiking transition vector.
\end{abstract}


\section{Introduction}

Membrane computing was initiated by P\u{a}un \cite{cell1} and has
developed very rapidly (already in 2003, ISI considered membrane
computing as ``fast emerging research area in computer science", see
{\tt http://esi-topics.com}).  It aims to abstract computing ideas
(data structures, operations with data, computing models, etc.) from
the structure and the functioning of single cell and from complexes
of cells, such as tissues and organs, including the brain. The
obtained models are distributed and parallel computing devices,
called \emph{P systems}. For updated information about membrane
computing, please refer to \cite{hMC}.

This work deals with a class of neural-like P systems, called
\emph{spiking neural P systems} (SN P systems, for short)
\cite{Ionescu1}. SN P systems were inspired by the
neurophysiological behavior of neurons (in brain) sending electrical
impulses along axons to other neurons, with the aim of incorporating
specific ideas from spiking neurons into membrane computing.
Generally speaking, in an SN P system the processing elements are
called \emph{neurons} and are placed in the nodes of a directed
graph, called the \emph{synapse graph}. The content of each neuron
consists of a number of copies of a single object type, namely the
\emph{spike}. Each neuron may also contain rules which allow to
remove a given number of spikes from it, or send spikes (possibly
with a delay) to other neurons. The application of every rule is
determined by checking the content of the neuron against a regular
set associated with the rule.

Representation of P systems by discrete structures has been one
topic in the field of membrane computing. One of the promising
discrete structures to represent P systems is matrix. Models with
matrices as their representation have been helpful to physical
scientists -- biologists, chemists, physicists, engineers,
statisticians, and economists -- solving real world problems.
Recently, matrix representation was introduced for represent a
restricted form of cell-like P systems without dissolution (where
only non-cooperative rules are used) \cite{Miguel}. It was proved
that with an algebraic representation P systems can be easily
simulated and computed backward (that is, to find all the
configurations that produce a given one in one computational step).

In this work, a matrix representation of SN P systems without delay
is proposed, where configuration vectors are defined to represent
the number of spikes in neurons; spiking vectors are used to denote
which rules will be applied; a spiking transition matrix is used to
describe the skeleton of a system; the transition net gain vectors are
also introduced to quantify the total amount of spikes consumed and
produced after the chosen rules are applied. With this algebraic
representation, matrix transition can be used to compute the next
configuration from a given one.

In addition, we consider another variant of SN P systems, SN P
systems with weights (WSN P systems, for short), which were
introduced in \cite{WSNP}. In these systems, each neuron has a
potential and a given threshold, whose values are real (computable)
numbers. Each neuron fires when its potential is equal to its
threshold; at that time, part of the potential is consumed and a
unit potential is produced (a spike). The unit potential is passed
to neighboring neurons multiplied with the weights of synapses. The
weights of synapses can also be real (computable) numbers. This
variant of SN P system allows real numbers to be computed by the
system.

A matrix over $\mathbb R_C$ is defined to represent WSN P systems,
where vectors are defined to represent the potentials in neurons and
the application of rules. In particular, when the potential of a
neuron is smaller than its spiking threshold, then this potential
vanishes (the potential of the neuron is set to zero). Therefore, a
forgetting vector is introduced to denote the potential vanishing
from neurons. It is also shown that matrix representation is
convenient for deciding the next configuration of the system from a
given configuration.

The rest of this paper is organized as follows. In the next section,
the definition of SN P systems is introduced. In Section
\ref{define-Mat} the matrix representation of SN P systems is given,
and an example is given to illustrate how to represent the
computation of an SN P system by matrix transition. In Section
\ref{Mat-WTSNP} the matrix representation method is extended to WSN
P systems. Conclusions and remarks are given in Section 5.

\section{Spiking Neural P Systems}\label{definiton-SNP}

In this section, a restricted variant of SN P systems, SN P systems
without delay, is introduced.

\begin{definition} An SN P system without delay, of degree $m\geq 1$,
is a construct of the form
$$\Pi=(O,\sigma_1,\ldots, \sigma_m, syn, in, out),$$
where:
\begin{enumerate}
\item[1.] $O=\{a\}$ is the singleton alphabet ($a$ is called spike);

\item[2.] $\sigma_1,\ldots, \sigma_m$ are neurons, of the form
$$\sigma_{i}=(n_i, R_i),1\leq i\leq m,$$
where:
\begin{enumerate}
\item[a)] $n_i\geq 0$ is the initial number of spikes contained in $\sigma_i$;

\item[b)] $R_i$ is a finite set of rules of the following two forms:
\begin{enumerate}
\item[(1)]$E/a^c \rightarrow a^p$, where $E$ is a regular expression
over $a$, and $c\geq 1$, $p\geq 1$, with the restriction $c\geq p$;
\item[(2)]$a^s\rightarrow \lambda$, for $s\geq 1$, with the
restriction that for each rule $E/a^c\rightarrow a^p$ of type (1)
from $R_i$, $a^s\notin L(E)$;
\end{enumerate}
\end{enumerate}
\item[3.] $syn= \{ (i,j)\, |\, 1\leq i,j \leq m, \, i\neq j\, \}$ (synapses
between neurons);

\item[4.] $in, out\in \{1,2,\ldots, m\}$ indicate the input and output neurons, respectively.
\end{enumerate}

\end{definition}

The rules of type (1) are spiking (also called firing) rules, which
are applied as follows. If the neuron $\sigma_i$ contains $k$
spikes, and $a^k \in L(E), k\geq c$, then the rule $E/a^c
\rightarrow a^p \in R_i$ can be applied. This means that consuming
(removing) $c$ spikes (thus only $k-c$ spikes remain in $\sigma_i$),
the neuron is fired, and it produces $p$ spikes; these spikes are
transported to all neighbor neurons by outgoing synapses. If a rule
$E/a^c\rightarrow a^p$ has $E=a^c$, then it is written in the
simplified form $a^c\rightarrow a^p$.

The rules of type (2) are forgetting rules; they are applied as
follows: if the neuron $\sigma_i$ contains exactly $s$ spikes, then
the rule $a^s\rightarrow \lambda$ from $R_i$ can be used, meaning
that all $s$ spikes are removed from $\sigma_i$.


In each time unit, if a neuron $\sigma_i$ can apply one of its
rules, then a rule from $R_i$ must be applied. Since two spiking
rules, $E_1/a^{c_1}\rightarrow a^{p_1}$ and $E_2/a^{c_2}\rightarrow
a^{p_2}$, can have $L(E_1) \cap L(E_2)\neq \emptyset$, it is
possible that two or more rules are applicable in a neuron. In that
case, only one of them is chosen and applied non-deterministically.
However, note that, by definition, if a spiking rule is applicable,
then no forgetting rule is applicable, and vice versa.

Thus, the rules are applied in the sequential manner in each neuron,
at most one in each step, but neurons function in parallel with each
other. It is important to notice that the applicability of a rule is
established based on the total number of spikes contained in the
neuron.

The configuration of the system is described by the number of spikes
present in each neuron. Using the rules as described above, one can
define transitions among configurations. Any sequence of transitions
starting in the initial configuration is called a computation. A
computation halts if it reaches a configuration where no rule can be
applied. The result of a computation is the number of steps elapsed
between the first two spikes sent by the output neuron to the
environment during the computation.

\section{Matrix Representation of SN P Systems}\label{define-Mat}

In this section, a matrix representation of SN P systems is given.
Based on this representation, it is shown how the computation of SN
P systems can be represented by operations with matrices.

As mentioned in the above section, a configuration of the system is
described by the number of spikes present in each neuron. Here,
vectors are used to represent configurations.

\begin{definition}[Configuration Vectors]\label{def_config_vec}
Let $\Pi$ be an SN P system with $m$ neurons, the vector $C_{0} =
(n_{1}, n_{2}, \ldots , n_{m})$ is called the {\bf initial
configuration vector} of $\Pi$, where $n_{i}$ is the amount of the
initial spikes present in neuron $\sigma_i$, $i=1,2, \ldots , m$
before a computation starts.

In a computation, for any $k \in \mathbb N$, the vector $C_{k} =
(n_{1}^{(k)}, n_{2}^{(k)}, \ldots , n_{m}^{(k)})$ is called the {\bf
$k$th configuration vector} of the system, where $n_{i}^{(k)}$ is
the amount of spikes in neuron $\sigma_i$, $i=1,2, \ldots, m$ after
the $k$th step of the computation.
\end{definition}

In order to describe which rules are chosen and applied in each
configuration, {\it spiking vectors} are defined.

\begin{definition}[Spiking Vectors] \label{def_spik_vec}
Let $\Pi$ be an SN P system with $m$ neurons and $n$ rules, and
$C_{k} = (n_{1}^{(k)}, n_{2}^{(k)}, \ldots , n_{m}^{(k)})$ be the
$k$th configuration vector of $\Pi$. Assume a total order
$d:1,\dots,n$ is given for all the $n$ rules, so the rules can be
referred as $r_1,\dots, r_n$. A {\bf spiking vector} $s^{(k)}$ is
defined as follows:
$$s^{(k)} = (r_1^{(k)}, r_2^{(k)}, \ldots , r_n^{(k)}),$$
where:
$$
r_{i}^{(k)}= \left\{ \begin{array}{rl}
 1, &\mbox{if the regular expression $E_{i}$ of rule $r_{i}$ is satisfied by the}  \\
  & \mbox {number of spikes $n_{j}^{(k)}$ (rule $r_{i}$ is in neuron $\sigma_j$) and }\\
  & \mbox {rule $r_i$ is chosen and applied};\\
 0, &\mbox{otherwise}.
       \end{array} \right.
$$

In particular, $s^{(0)} = (r_1^{(0)}, r_2^{(0)}, \ldots ,
r_n^{(0)})$ is called the {\bf initial spiking vector}.
\end{definition}

The application of each rule will change the number of spikes in
some neurons, for example, when the rule $r_i:$ $E/a^c\rightarrow
a^p$ is applied in neuron $\sigma_j$, it consumes $c$ spikes in
$\sigma_j$, and emits $p$ spikes; these $p$ spikes are immediately
delivered to all the neurons $\sigma_s$ such that $(j,s)\in syn$.
Here, a \emph{spiking transition matrix} is defined to denote the
amount of spikes consumed (or received) by each neuron via each
rule.

\begin{definition} [Spiking Transition Matrix]\label{defi-snp-mat}
Let $\Pi$ be an SN P system with $m$ neurons and $n$ rules, and
$d:1,\dots, n$ be a total order given for all the $n$ rules. The
{\bf spiking transition matrix} of the system $\Pi$, $M_{\Pi}$, is
defined as follows:


$$M_{\Pi} = [a_{ij}]_{n \times m},$$
where:
$$
a_{ij} = \left\{
\begin{array}{rl}
-c, &\mbox{if rule $r_i$ is in neuron $\sigma_j$ and it is applied consuming $c$ spikes;} \\
 p, &\mbox{if rule $r_i$ is in neuron $\sigma_s$ ($s\neq j$ and $(s,j)\in syn$)} \\
 & \mbox{and it is applied producing $p$ spikes;}\\
 0, &\mbox{if rule $r_i$ is in neuron $\sigma_s$ ($s\neq j$ and $(s,j)\notin syn$).}
    \end{array}
\right.
$$
\end{definition}

In a spiking transition matrix, the row $i$ is associated with the
rule $r_i:$ $E/a^c\rightarrow a^p$. Assume that the rule $r_i$ is in
neuron $\sigma_j$. When the rule $r_i$ is applied, it consumes $c$
spikes in neuron $\sigma_j$; neuron $\sigma_s$ ($s\neq j$ and
$(j,s)\in syn$) receives $p$ spikes from neuron $\sigma_j$; neuron
$\sigma_s$ ($s\neq j$ and $(j,s)\notin syn$) receives no spike from
neuron $\sigma_j$. By the definition of spiking transition matrix,
the entry in the position $(i,j)$ is a negative number; the other
entries in the row $i$ are non-negative numbers. So the following
observation holds:

\smallskip
\textbf{Observation 1:} each row of a spiking transition matrix has
exactly one negative entry.
\smallskip

In a spiking transition matrix, the column $i$ is associated with
neuron $\sigma_i$.
 For an SN P system, without loss of generality,
 it can be assumed that each neuron has at least one rule inside
 (if a neuron has no rule inside, it just stores spikes, sending no spikes to other neurons or environment,
 so it can be deleted without any influence to
 the computational result of the system). Assume the rules in neuron $\sigma_i$ are $r_m$, $r_n, \dots$. These rules consume
 spikes of neuron $\sigma_i$ when they are applied. So the corresponding entries $(m,i), (n,i), \dots$ in the spiking
 transition matrix are negative
 numbers, and the following observation holds:

\smallskip
\textbf{Observation2:} each column of a spiking transition matrix
has at least one negative entry.
\smallskip

In the following, it will be shown that how matrices representing SN
P systems can be used to represent the computation of SN P systems
by operating with matrices. Before the matrix operations for SN P
systems are formally defined, a simple example is provided as
follows.
\begin{example}\label{ex1}
Let us consider an SN P system $\Pi=(\{a\},\sigma_1,\sigma_1,
\sigma_3, syn, out)$ that generates the set $\mathbb N$ of natural
numbers excluding $1,$ where $\sigma_1=(2,R_1)$, with
$R_1=\{a^2/a\rightarrow a, a^2\rightarrow a\}$; $\sigma_2=(1,R_2)$,
with $R_2=\{a\rightarrow a\}$; $\sigma_3=(1,R_3)$, with
$R_3=\{a\rightarrow a, a^2\rightarrow \lambda\}$;
$syn=\{(1,2),(1,3),(2,1),(2,3)\}$; $out=3$. $\Pi$ is also
represented graphically in Figure \ref{fig-ex1}, which may be easier
to understand.
\begin{figure}[htbp]
\centering
\includegraphics[width=68mm]{example.eps}
          \caption{An SN P system $\Pi$ that generates the set $\mathbb N - \{1\}$}
          \label{fig-ex1}
       \end{figure}
\end{example}
In order to represent the above SN P system $\Pi$ in a matrix, a
total order is set for all the rules in the system, which can be
seen in Figure \ref{fig-ex1}. With this order, the rules can be
denoted by $r_1,\dots, r_5$.

Let $M_{\Pi_1}=[a_{ij}]_{5 \times 3}$ be the spiking transition
matrix for $\Pi$. By Definition \ref{defi-snp-mat}, the row $i$ of
$M_{\Pi}$ is associated with the rule $r_i: E/a^c\rightarrow a^p,
c\geq 1,p\geq 0$ in system $\Pi$. The entries $a_{i1},a_{i2},a_{i3}$
are the amount of spikes which neurons $\sigma_1,\sigma_2,\sigma_3$
will get (or consume) when rule $r_i$ is applied.

%If $r_i$ is in neuron $\sigma_j$, we define the entries $a_{ij}$ as
%the negative of the amount of spikes which rule $r_{i}$ will consume
%when it is applied in the computation. We put $p$ for $a_{ij}$ if
%and only if there is a synapse from the neuron that contains rule
%$r_i,$ say neuron $\sigma_s$ to neuron $\sigma_j$. $a_{ij}$ is $0$
%implies there will be no change in neuron $j$ when we apply rule
%$r_{i}.$

The spiking transition matrix for the SN P system $\Pi$ depicted
 in Figure \ref{fig-ex1} is as follows.\\
[2mm]
%\begin{center}
\begin{equation}\label{snp_mat}
%$$
M_{\Pi} = \left( \begin{array}{ccc}
  -1 & 1 & 1\\
  -2 & 1 &  1 \\
   1 &  -1 & 1 \\
   0 & 0 & -1\\
   0 & 0& -2
  \end{array}\right)
%$$
\end{equation}
%\end{center}
\\
[0.5mm]


 Initially, neurons $\sigma_1$, $\sigma_2$, and $\sigma_3$ have
  2, 1, and 1 spike(s), respectively. According to Definition \ref{def_config_vec},
  the initial configuration vector
 for system $\Pi$ would be $C_{0} = (2,1,1)$.
 Since neuron $\sigma_1$ has two rules
 $r_{1}$ and $r_{2}$ that are applicable in the
 initial transition, one of the rules could be chosen, the initial
 spiking transition vector would be $(1,0,1,1,0)$ or $(0,1,1,1,0)$ by Definition \ref{def_spik_vec}.
  Note that the rule $r_{5}$ is not applicable because the regular expression
 $a^{2}$ is not satisfied in neuron $\sigma_3$.

If the rule $r_1:a^2/a\rightarrow a$ is applied, it consumes one
spike in neuron $\sigma_1$ and sends one spike to neurons $\sigma_2$
and $\sigma_3$, respectively; at the same time, neuron $\sigma_2$
sends one spike to each of the neurons $\sigma_1$ and $\sigma_3$. In
this step, the net gain of neuron $\sigma_1$ is 0 spike (it consumes
1 spike by $r_1$ and receives 1 spike from neuron $\sigma_2$); the
net gain of neuron $\sigma_2$ is 0 spike (it consumes 1 spike by
$r_3$ and receives 1 spike from neuron $\sigma_1$); the net gain of
neuron $\sigma_3$ is 1 spike (it consumes 1 spike by rule $r_5$ and
receives 1 spike from each of the neurons $\sigma_1$ and
$\sigma_2$). After this step, the numbers of spikes in neurons
$\sigma_1$, $ \sigma_2$ and $\sigma_3$ are 2, 1 and 2, respectively.

The illustration above explained intuitively how an SN P system
compute from one configuration to the succeeding one. In order to
use matrix operations to represent it, the following definitions are
needed:

\begin{definition}[Transition Net Gain Vector]
Let $\Pi$ be an SN P system with $m$ neurons and $n$ rules, and
$C_{k} = (n_{1}^{(k)}, n_{2}^{(k)}, \ldots , n_{m}^{(k)})$ be the
$k$th configuration vector of $\Pi$. The {\bf transition net gain
vector} at step $k$ is defined as $NG^{(k)}= C_{k+1}~ - ~ C_{k}.$
\end{definition}


\begin{lemma}
Let $\Pi$ be an SN P system with $m$ neurons and $n$ rules,
$d:1,\dots, n$ be a total order for the $n$ rules, $M_{\Pi}$ the
spiking transition matrix of $\Pi$, and $s^{(k)}$ the spiking vector
at step $k$. Then the transition net gain vector at step $k$ can be
obtained by
\begin{equation}\label{transition-gain}
NG^{(k)} = s^{(k)}\cdot M_{\Pi}.
\end{equation}
\end{lemma}
\begin{proof} Assume that $M_{\Pi}=[a_{ij}]_{m\times n}$, $s^{(k)} = (r_1^{(k)},
r_2^{(k)}, \ldots , r_n^{(k)})$, and $NG^{(k)}= (g_{1}, g_{2},
\ldots,$ $g_{m})$. Note that the spiking vector $s^{(k)}$ is a
$\{0,1\}$-vector that identifies the rules that would be applied at
step $k$. Thus, $\Sigma_{i=1}^{n}\, r_{i}^{(k)}a_{ij}$ represents
the total amount of spikes received and consumed by neuron
$\sigma_{j}$ after applying the rules identified by $s^{(k)}$.
Therefore, the net gain of neuron $\sigma_j$ is
$g_{j}=\Sigma_{i=1}^{n}\, r_{i}^{(k)}a_{ij}$, for all $j = 1,2,
\ldots ,m$. That is, $NG^{(k)} = s^{(k)}\cdot M_{\Pi}$.
\end{proof}

\begin{theorem}
Let $\Pi$ be an SN P system with $m$ neurons and $n$ rules,
$d:1,\dots, n$ be a total order for the $n$ rules, $M_{\Pi}$ the
spiking transition matrix of $\Pi$, $C_k$ the kth configuration
vector, and $s^{(k)}$ the spiking vector at step $k$, then the
configuration $C_{k+1}$ of $\Pi$ can be obtained by
\begin{equation}\label{next-config}
C_{k+1} =  C_{k} ~ + ~ s^{(k)}\cdot M_{\Pi}.
\end{equation}
\end{theorem}
\begin{proof}
This results follows directly from the preceding Lemma.
\end{proof}

Let us go back to the example shown in Figure \ref{fig-ex1}. Given
the initial configuration vector $C_0=(2,1,1)$, the next
configuration of system $\Pi$ can be computed as follows.

If the rules $r_1,r_3,r_4$ are chosen to be applied, the spiking
vector is $s^{(0)}=(1,0,1,1,0)$, and the next configuration is
\begin{equation}
C_1=(2, 1, 1)~+~(1, 0, 1, 1, 0)\left(
\begin{array}{ccc}
 -1 & 1 & 1 \\
 -2 & 1 & 1 \\
 1 & -1 & 1\\
 0 & 0 & -1\\
 0 & 0 & -2
\end{array}
\right)=(2, 1, 2).
\end{equation}

In the next step, $r_1, r_3, r_5$ are chosen to be applied, the
spiking vector is $(1,0,1,0,1)$, and the next configuration is
\begin{equation}
C_2=(2, 1, 2)~+~(1, 0, 1, 0, 1)\left(
\begin{array}{ccc}
 -1 & 1 & 1 \\
 -2 & 1 & 1 \\
 1 & -1 & 1\\
 0 & 0 & -1\\
 0 & 0 & -2
\end{array}
\right)=(2, 1, 2),
\end{equation}
where the transition net gain vector is
\begin{equation} \label{NG=0}
NG=(1, 0, 1, 0, 1)\left(
\begin{array}{ccc}
 -1 & 1 & 1 \\
 -2 & 1 & 1 \\
 1 & -1 & 1\\
 0 & 0 & -1\\
 0 & 0 & -2
\end{array}
\right)=(0, 0, 0).
\end{equation}

Equation (\ref{NG=0}) means that the configuration of the system
remains unchanged as long as the rules $r_1$, $r_3$ and $r_5$ are
chosen to be applied. However, at any moment, starting from the
first step of computation, neuron $\sigma_1$ can choose to apply the
rule $r_2:a^2\rightarrow a$. In this case, system will go to another
configuration. The checking is left to the readers.

The following Corollary is a direct consequence of the preceding
Theorem.

\begin{corollary}
Let $\Pi$ be an SN P system with $m$ neurons and $n$ rules,
$d:1,\dots, n$ be a total order for the $n$ rules, $M_{\Pi}$ the
spiking transition matrix of $\Pi$, $C_k$ the $k$th configuration
vector, and $s^{(k-1)}$ the spiking vector at step $k-1$, then the
previous configuration $C_{k-1}$ is
\begin{equation}\label{next-config}
C_{k-1} =  C_{k} ~ -  ~ s^{(k-1)}\cdot M_{\Pi}.
\end{equation}
\end{corollary}

In the above matrix representation, the spikes sent to the
environment are not considered. In order to consider the spikes sent
to the environment, an {\it augmented spiking transition matrix} is
introduced.

\begin{definition}[Augmented Transition Spiking Matrix]
Let $\Pi$ be an SN P system with $m$ neurons and $n$ rules,
$d:1,\dots, n$ be a total order for the $n$ rules, and $M_{\Pi}$ the
$n \times m$ spiking transition matrix of $\Pi$. An {\bf augmented
spiking transition matrix} is defined as
$$[M_{\Pi}\, |\, e]_{n \times (m+1)},$$\\
where the column $e=(e_1, e_2, \dots, e_n)^T$ represents the spikes
sent to the environment, with:
$$
e_{i} = \left\{
\begin{array}{rl}
 p, &\mbox{ \rm if rule $r_i$ is in the output neuron and it is applied producing $p$ spikes; } \\
 0, &\mbox{ \rm if rule $r_i$ is not in the output neuron. } \\
    \end{array}
\right.
$$
\end{definition}

Correspondingly, the {\bf augmented configuration vector} after the
$k$th step in the computation is defined as
$$C_{k} = (n_{1}^{(k)}, n_{2}^{(k)}, \ldots , n_{m}^{(k)},
n_{e}^{(k)}),$$ where $n_{i}^{(k)}$ is the amount of spikes in
neuron $\sigma_i$, for all $i=1,2, \ldots, m$, $n_{e}^{(k)}$ is the
amount of spikes collected in the environment. Using this vector
instead of the configuration vector in Definition
\ref{def_config_vec} allows us to monitor the output of the system.

\section{Matrix Representation for WSN P Systems}\label{Mat-WTSNP}

In this section, matrix representation for spiking neural P systems
with weights (WSN P systems, for short) is investigated. Instead of
counting spikes as in usual SN P systems, each neuron in WSN P
systems contains a potential, which can be expressed by a computable
real number. Each neuron fires when its potential is equal to the
given threshold. The execution of a rule consumes part of the
potential and produces a unit potential. This unit potential passes
to neighboring neurons multiplied with the weights of synapses. In
an SN P system with weights, the involved numbers -- weights,
thresholds, potential consumed by each rule -- can be real
(computable) numbers. Formally, the system is defined as follows.

\begin{definition} An SN P system with weights, of degree $m\geq 1$,
is a construct of the form
$$\Pi=(\sigma_1,\ldots, \sigma_m, syn, in, out),$$
where:
\begin{enumerate}
\item[1.] $\sigma_1,\ldots, \sigma_m$ are neurons of the form
$\sigma_{i}=(p_i, R_i),1\leq i\leq m,$ where:
\begin{enumerate}
\item[a)] $p_i\in \mathbb{R}_c$ is the initial potential in $\sigma_i$;

\item[b)] $R_i$ is a finite set of rules of the form
$T_i/d_s \rightarrow 1$, $s=1,2,\dots,n_i$ for some $n_i\geq 1$,
where $T_i\in \mathbb{R}_c$, $T_i\geq 1$, is the firing threshold
potential of neuron $\sigma_i$, and $d_s\in \mathbb{R}_c$ with the
restriction $0<d_s\leq T_i$;
\end{enumerate}
\item[2.] $syn\subseteq \{1,2,\ldots,m\}\times \{1,2,\ldots,m\}\times
\mathbb{R}_c$ are synapses between neurons, with $i\neq j$, $w\neq
0$ for each $(i,j,w)\in syn, 1\leq i,j \leq m$;

\item[3.] $in, out\in \{1,2,\ldots, m\}$ indicate the input and output neurons, respectively.
\end{enumerate}

\end{definition}

The spiking rules are applied as follows. Assume that at a given
moment, neuron $\sigma_i$ has the potential equal to $p$. If $p =
T_i$, then any rule $T_i/d_s \rightarrow 1 \in R_i$ can be applied.
The execution of this rule consumes an amount of $d_s$ of the
potential (thus leaving the potential $T_i-d_s$) and prepares one
unit potential (we also say a spike) to be delivered to all the
neurons $\sigma_j$ such that $(i,j,w)\in syn$. Specifically, each of
these neurons $\sigma_j$ receives a quantity of potential equal to
$w$, which is added to the existing potential in $\sigma_j$. Note
that $w$ can be positive or negative, hence the potential of the
receiving neuron is increased or decreased. The potential emitted by
a neuron $\sigma_i$ passes immediately to all neurons $\sigma_j$
such that $(i,j,w)\in syn$, that is, the transition of potential
takes no time. If a neuron $\sigma_i$ spikes and it has no outgoing
synapse, then the potential emitted by neuron $\sigma_i$ is lost.

The main feature of this system is: (1) each neuron $\sigma_i$ has
only one fixed threshold potential $T_i$; (2) if a neuron has the
potential equal to its threshold potential, then all rules
associated with this neuron are enabled, and only one of them is
non--deterministically chosen to be applied; (3) when a neuron
spikes, there is always only one unit potential (a spike) emitted.

If neuron $\sigma_i$ has the potential $p$ such that $p < T_i$, then
the neuron $\sigma_i$ returns to the resting potential 0. If neuron
$\sigma_i$ has the potential $p$ such that $p > T_i$, then the
potential $p$ keeps unchanged.

Summing up, if neuron $\sigma_i$ has potential $p$ and receives
potential $k$ at step $t$, then at step $t+1$ it has the potential
$p'$, where:


\begin{displaymath}
p'= \left\{ \begin{array}{ll} k, & \textrm{if $p<T_i$;}\\
p-d_s+k, & \textrm{if $p=T_i$ and rule $T_i/d_s\rightarrow 1$ is applied;} \\
$p+k$, & \textrm{if $p> T_i$.}
\end{array} \right.
\end{displaymath}

The configuration of the system is described by the distribution of
potentials in neurons. Using the rules described above, one can
define transitions among configurations. Any sequence of transitions
starting in the initial configuration is called a computation. A
computation halts if it reaches a configuration where no rule can be
applied. The result of a computation is the number of steps elapsed
between the first two spikes sent by the output neuron to the
environment during the computation.

Similar to Section \ref{definiton-SNP}, some vectors are defined to
represent configurations and the application of rules.

\begin{definition}[Configuration Vectors]
Let $\Pi$ be a WSN P system with $m$ neurons, the vector $C_{0} =
(p_{1}, p_{2}, \ldots , p_{m})$ is called the {\bf initial
configuration vector} of $\Pi$, where $p_{i}$ is the amount of the
initial potential present in neuron $\sigma_i$, $i=1,2, \ldots , m$
before a computation starts.

In a computation, for any $k \in \mathbb N$, the vector $C_{k} =
(p_{1}^{(k)}, p_{2}^{(k)}, \ldots , p_{m}^{(k)})$ is called the {\bf
$k$th configuration vector} of the system, where $p_{i}^{(k)}$ is
the amount of spikes in neuron $\sigma_i$, $i=1,2, \ldots, m$ after
the $k$th step in the computation.
\end{definition}

\begin{definition}[Spiking Vectors]\label{SV-WTSNP}
Let $\Pi$ be a WSN P system with $m$ neurons and $n$ rules, and
$C_{k} = (p_{1}^{(k)}, p_{2}^{(k)}, \ldots , p_{m}^{(k)})$ be the
kth configuration vector of $\Pi$. Assume a total order
$d:1,\dots,n$ is given for all the $n$ rules, so the rules can be
referred as $r_1,\dots, r_n$. A {\bf spiking vector} $s^{(k)}$ is
defined as follows:
$$s^{(k)} = (r_1^{(k)}, r_2^{(k)}, \ldots , r_n^{(k)}),$$
where:
$$
r_{i}^{(k)}= \left\{ \begin{array}{rl}
 1, &\mbox{if $r_i$ in neuron $\sigma_j$ is enabled (the potential $p_j^{(k)}$ in neuron $\sigma_j$ is equal} \\
  & \mbox { to its spiking threshold $t_j$) and the rule $r_i$ is chosen and applied};\\
 0, &\mbox{otherwise}.
       \end{array} \right.
$$
In particular, $s^{(0)} = (r_1^{(0)}, r_2^{(0)}, \ldots ,
r_n^{(0)})$ is called the {\bf initial spiking vector}.
\end{definition}

In WSN P systems, when the potential of a neuron is smaller than its
spiking threshold, then this potential vanishes, the potential of
the neuron is set to zero. In order to describe which neurons forget
their potentials, \emph{forgetting vector} is defined.

\begin{definition}[Forgetting vector]
Let $\Pi$ be a WSN P system with $m$ neurons, and $C_{k} =
(p_{1}^{(k)}, p_{2}^{(k)}, \ldots , p_{m}^{(k)})$ be the kth
configuration vector of $\Pi$. A {\bf forgetting vector}
$forg^{(k)}$ is defined as follows:
$$forg^{(k)} = (f_1^{(k)}, f_2^{(k)}, \ldots , f_m^{(k)}),$$
where:
$$
f_{i}^{(k)}= \left\{ \begin{array}{rl}
 1 &\mbox{ if the potential $p_j^{(k)}$ in neuron $j$ is less than its spiking threshold $t_j$;} \\
 0 &\mbox{ otherwise.}
       \end{array} \right.
$$
In particular, $forg^{(0)} = (f_1^{(0)}, f_2^{(0)}, \ldots ,
f_m^{(0)})$ is called the {\bf initial forgetting vector}.

\end{definition}

For a WSN P system, a \emph{spiking transition matrix} is defined in
order to store the information of the amount of potential consumed
(or received) by each neuron when each rule is applied.

\begin{definition} [Spiking Transition Matrix]\label{defi-wtsnp-mat}
Let $\Pi$ be a WSN P system with $m$ neurons and $n$ rules,
$d:1,\dots, n$ a total order given for all the $n$ rules. The {\bf
spiking transition matrix} of the system $\Pi$, $M_{\Pi}$, is
defined as follows:


$$M_{\Pi} = [a_{ij}]_{n \times m},$$
where:
$$
a_{ij} = \left\{
\begin{array}{rl}
-c, &\mbox{if rule $r_i$ is in neuron $\sigma_j$ and it is applied consuming potential $c$;} \\
 w, &\mbox{if rule $r_i$ is in neuron $\sigma_s$ ($s\neq j$ and $(s,j,w)\in syn$)} \\
 & \mbox{and it is applied;}\\
 0, &\mbox{if rule $r_i$ is in neuron $\sigma_s$ ($s\neq j$ and $(s,j,w)\notin syn$, for $w\in \mathbb{R}_c$).}
    \end{array}
\right.
$$
\end{definition}

The following example illustrates how to get the matrix
representation of a WSN P system.

\begin{example}\label{ex2}
Let us consider a WSN P system $\Pi_1=(\sigma_1,\sigma_1, \sigma_3,
syn, out)$ that generates the set $\mathbb N$ of natural numbers
excluding $1$ and $2$, where $\sigma_1=(3,R_1)$, with
$R_1=\{1.5/1.5\rightarrow 1, 1.5/1\rightarrow 1\}$;
$\sigma_2=(2,R_2)$, with $R_2=\{1/1 \rightarrow 1\}$;
$\sigma_3=(1.5,R_3)$, with $R_3=\{1.5/1\rightarrow 1\}$;
$syn=\{(1,2,1)$,$(1,3,-0.5),(2,1,1)$,$(2,3,1.5)$,\\$(3,1,-1.5)$,$(3,2,-1)\}$;
$out=3$. $\Pi_1$ is also represented graphically in Figure
\ref{fig-ex2}. Note that in Figure \ref{fig-ex2}, when the weight on
a synapse is one, it is omitted.
\begin{figure}[h]
          \centering
\includegraphics[width=68mm]{example2.eps}
          \caption{A WSN P system $\Pi_1$ that generates the set $\mathbb N - \{1,2\}$}
          \label{fig-ex2}
       \end{figure}
\end{example}


As shown in Figure \ref{fig-ex2}, a total order for the four rules
is set. With this order, the rules can be referred as $r_1$, $r_2$,
$r_3$, $r_4$. According to Definition \ref{defi-wtsnp-mat}, the
spiking transition matrix $M_{\Pi_1}$ for the
 WSN P system $\Pi_1$ is
\begin{equation}\label{wtsnp_mat}
M_{\Pi_1} = \left( \begin{array}{ccc}
  -1.5 & 1 & -0.5\\
  -1   & 1 &-0.5\\
  1.5  & -1 & 1.5\\
  -1.5 & -1 &  -1.5 \\
  \end{array}\right)
\end{equation}

The initial configuration is $C_{0} = (3,2,1.5)$. The initial
spiking vector is $(0,0,0,1)$ by Definition \ref{SV-WTSNP} and the
order of rules.


At step 1, only the output neuron $\sigma_3$ spikes, while the other
two neurons $\sigma_1$, $\sigma_2$ maintain their potentials,
because their potentials are greater than their corresponding firing
thresholds. Neurons $\sigma_1$ and $\sigma_2$ receive potentials
$-1.5$ and $-1$, respectively. After this step, the configuration
vector becomes $C_{1} = (1.5,1,0.5).$ At step 2, neurons $\sigma_1$
and $\sigma_2$ have potentials 1.5 and 1, respectively, which equal
to their corresponding firing thresholds, hence both neurons
$\sigma_1$ and $\sigma_2$ spike. Neuron $\sigma_1$ has two rules
$r_1:1.5/1.5\rightarrow 1$ and $r_2:1.5/1\rightarrow 1$, and one of
them is non-deterministically chosen. If rule $r_1$ is chosen to be
applied, it consumes potential 1.5 and, at the same time, it
receives 1.5 unit of potential from neuron $\sigma_2$. Hence, in
this step, the net gain of potential in neuron $\sigma_1$ is 0 and
at the next step neuron $\sigma_1$ still has potential 1.5. The net
gain of potential in neuron $\sigma_2$ is also 0 (one unit of
potential is consumed by $r_3$ and one unit of potential is received
from neuron $\sigma_1$), neuron $\sigma_3$ forgets its potential 0.5
(because it is less than the threshold 1.5), but gets another
potential 0.5 from the other two neurons (receives potential $-0.5$
from $\sigma_1$ and potential 1 from $\sigma_2$). At the next step,
the numbers of spikes in neurons $\sigma_1$, $\sigma_2$, and
$\sigma_3$ are still 1.5, 1 and 0.5, respectively.

In order to denote the change of numbers of spikes in each neuron, a
\emph{transaction net gain vector} is defined.

\begin{definition}[Transition Net Gain Vector]
Let $\Pi$ be a WSN P system with $m$ neurons and $n$ rules, and
$C_{k} = (p_{1}^{(k)}, p_{2}^{(k)}, \ldots , p_{m}^{(k)})$ be the
$k$th configuration vector of $\Pi$. The {\bf transition net gain
vector} at step $k$ is defined as $NG^{(k)}= C_{k+1} -  C_{k}$.
\end{definition}

In order to compute the transition net gain vector, the
\emph{Hadamard product} is used.

\begin{definition}[Hadamard
product] Let $A$ and $B$ be $m\times n$ matrices. The Hadamard
Product of $A$ and $B$ is defined by $[A\odot B]_{ij}=A_{ij}B_{ij}$
for all $1\leq i\leq m$, $1\leq j\leq n$.

\end{definition}

To avoid confusion, juxtaposition of matrices will imply the
``usual'' matrix multiplication, and ``$\odot$'' is used for the
Hadamard product.

\begin{lemma}
Let $\Pi$ be a WSN P system with $m$ neurons and $n$ rules,
$d:1,\dots, n$ be a total order for the $n$ rules, $M_{\Pi}$ the
spiking transition matrix of $\Pi$, $C_k$ the $k$th configuration
vector, $s^{(k)}$ the spiking vector at step $k$, and $forg^{(k)}$
the forgetting vector at step $k$. Then the transition net gain
vector at step $k$ is
\begin{equation}\label{transition-gain-wtsnp}
NG^{(k)} = s^{(k)}\cdot M_{\Pi}-forg^{(k)}\odot C_{k}.
\end{equation}
\end{lemma}

\begin{proof}
Assume that $M_{\Pi}=[a_{ij}]_{m\times n}$, $C_{k} = (p_{1}^{(k)},
p_{2}^{(k)}$, $\ldots, p_{m}^{(k)})$, $s^{(k)} = (r_1^{(k)},
r_2^{(k)}$, $\ldots,$ $r_n^{(k)})$,  $forg^{(k)} = (f_1^{(k)},
f_2^{(k)}, \ldots , f_m^{(k)})$, and $NG^{(k)}= (g_{1}, g_{2},
\ldots , g_{m})$.  Note that $r_i^{(k)}$ is a $\{0,1\}$-value that
identifies whether the rules $r_i$ would be applied, $f_i^{(k)}$ is
a $\{0,1\}$-value that identifies whether the neuron $\sigma_j$
would forget its potential. Thus, $\Sigma_{i=1}^{n}\,
r_{i}^{(k)}a_{ij}-f_{j}^{(k)}p_j^{(k)}$ represents the total amount
of potential obtained, consumed and forgotten by neuron $\sigma_{j}$
at the $k$th step. Therefore, the net gain of neuron $\sigma_j$ is
$g_{j}=\Sigma_{i=1}^{n}\, r_{i}^{(k)}a_{ij}-f_{j}^{(k)}p_j^{(k)},$
for all $j = 1,2, \ldots ,m$. That is, $NG^{(k)} = s^{(k)}\cdot
M_{\Pi}-forg^{(k)}\odot C_{k}.$
\end{proof}

\begin{theorem}
Let $\Pi$ be a WSN P system with $m$ neurons and $n$ rules,
$d:1,\dots, n$ be a total order for the $n$ rules, $M_{\Pi}$ the
spiking transition matrix of $\Pi$, $C_k$ the $k$th configuration
vector,  $s^{(k)}$ the spiking vector at step $k$, and $forg^{(k)}$
the forgetting vector at step $k$. Then the configuration $C_{k+1}$
of $\Pi$ can be obtained by
\begin{equation}\label{next-config}
C_{k+1} =  C_{k} ~ + ~ s^{(k)}\cdot M_{\Pi}-forg^{(k)}\odot C_{k}.
\end{equation}
\end{theorem}
\begin{proof}
This result follows directly from the preceding lemma.
\end{proof}

In Example \ref{ex2} shown in Figure \ref{fig-ex2}, $C_0=(3,2,1.5)$,
at step 1 only the rule $r_4$ is applicable. As the potentials in
all the neurons are higher than their threshold, no neuron forgets
its potential. Therefore, $s^{(0)}=(0,0,0,1)$, $forg^{(0)}=(0,0,0)$,
the next configuration can be obtained by
\begin{equation}
C_1=(3, 2, 1.5)+(0, 0,0,1)\left(
\begin{array}{ccc}
-1.5 & 1 & -0.5\\
  -1   & 1 &-0.5\\
  1.5  & -1 & 1.5\\
  -1.5 & -1 &  -1.5 \\
\end{array}
\right)-(0, 0, 0)\odot (3, 2, 1.5)
\end{equation}
That is, $C_1=(2, 1, 0.5)$.

In the next step, rules $r_1, r_3$ can be applied, so the spiking
vector is $(1,0,1,0)$. Because the potential in neuron $\sigma_3$ is
less than its threshold, neuron $\sigma_3$ forgets its potential,
and the forgetting vector is $forg^{(0)}=(0,0,1)$. Hence, the next
configuration is
\begin{equation}
C_2=(2, 1, 0.5)+(1, 0, 1, 0)\left(
\begin{array}{ccc}
-1.5 & 1 & -0.5\\
  -1   & 1 &-0.5\\
  1.5  & -1 & 1.5\\
  -1.5 & -1 &  -1.5 \\
\end{array}
\right)-(0, 0, 1)\odot (2, 1, 0.5)
\end{equation}
That is, $C_2=(2, 1, 0.5)$.

This example clearly shows how matrix representation and operation
can describe the computation of the system. Such matrix
representation is useful for the simulation of the system in
computer.

\section{Conclusions and Remarks}
In this work, an algebraic representation for SN P systems is
introduced. For every SN P system without delay, configuration
vectors are defined to represent the number of spikes in each
neuron; spiking vectors are used to denote which rules will be
applied; a spiking transition matrix is used to describe the
skeleton of system. Such algebraic representation is also extended
for another variant of SN P systems -- WSN P systems.

It is not difficult to see that such matrix representation is also
suitable for other variants of SN P systems, such as asynchronous SN
P systems \cite{Asynchronous} and SN P systems with exhaustive use
of rules \cite{Ionescu2}. The spiking transition matrix is related
to the structure of system only, so the elements of the matrix are
determined initially. During the computation of a system, it is only
needed to decide the spiking vector by checking the current
configuration vector and the regular expressions of rules. In
general, such algebraic representation is easy to be programmed for
computer simulation.

The systems considered in this paper have no delay, which
corresponds to the biological feature that neurons have refractory
time. It is open how to represent the computations of SN P systems
with delay by matrices.

\section*{Acknowledgement}
The work of X. Zeng and L. Pan was supported by National Natural
Science Foundation of China (Grant Nos. 60674106, 30870826 and
60703047) and HUST-SRF (2007Z015A). The work of  H. Adorna is
supported by Engineering Research and Development for Technology of
the DOST, Philippines. The work of M.A. Mart\'inez-del-Amor and M.J.
P\'erez-Jim\'enez is supported by the project TIN2009-13192 of the
Ministerio de Ciencia e Innovaci\'on of Spain, cofinanced by FEDER
funds, and the ``Proyecto de Excelencia con Investigador de
Reconocida Val\'ia'' of the Junta de Andaluc\'{i}a under grant
P08-TIC04200.

\nocite{*}
%\bibliographystyle{abbrv}
%\bibliography{Pcuda}

\begin{thebibliography}{99}

\bibitem{Asynchronous}
Cavaliere, M., Egecioglu, O., Ibarra, O.H., Woodworth, S., Ionescu,
M., P\u{a}un, G.: Asynchronous Spiking Neural P Systems. Theoretical
Computer Science 410, 2352--2364 (2009)

\bibitem{Miguel}
Guti\'errez-Naranjo, M.A., P\'{e}rez-Jim\'{e}nez M. J.: Searching
Previous Configurations in Membrane Computing. In: P\u{a}un, G.,
P��erez-Jim��enez, M.J., Riscos-N��u.nez, A. (eds.) Tenth Workshop
on Membrane Computing (WMC10). LNCS, vol. 5957, pp, 301--315,
Springer, Heidelberg (2010)

\bibitem{Ionescu1}
Ionescu, M., P\u{a}un, G., Yokomori, T.: Spiking Neural P Systems.
Fundamenta Informaticae 71(2--3), 279--308 (2006)

\bibitem{Ionescu2}
Ionescu, M., P\u{a}un, G., Yokomori, T.: Spiking Neural P Systems
with Exhaustive Use of Rules. International Journal of
Unconventional Computing 3, 135--154 (2007)

\bibitem{Mat-Structural}
Nelson, J.K., McCormac, J.C.: Structural Analysis: Using Classical
and Matrix Methods, 3rd Edition. Wiley (2003)

\bibitem{cell1}
P\u aun, G.: Computing with Membranes. Journal of Computer and
System Sciences 61(1), 108-143 (2000)

\bibitem{Gh}
P\u{a}un, G.: Membrane Computing -- An Introduction.
Springer-Verlag, Berlin (2002)

\bibitem{hMC} P\u{a}un, G., Rozenberg, G., Salomaa, A. (eds.)
Handbook of Membrane Computing. Oxford University Press (2010)

\bibitem{WSNP}
Wang, J., Hoogeboom, H.J., Pan, L., P\u{a}un, G., P\'erez-Jim\'enez
M.J.: Spiking Neural P Systems with Weights. Neural Computation (in
press)

\bibitem{Pweb}
The P System Web Page: \url{http://ppage.psystems.eu}

\end{thebibliography}

\end{document}
